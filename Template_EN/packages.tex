% -------------------------------------------------------

%   LaTeX日常文件模板 by Meng Zhiran

%   packages.tex : 宏包引用文件,请在不会引起宏包冲突的前提下增加或删除宏包

% -------------------------------------------------------

% -----------------宏包功能简介与使用手册------------------

% (1){geometry}包:       设置页面的左、右、上、下页边距 [https://texdoc.org/serve/geometry/0]
% (2){fancyhdr}包:       提供页眉页脚设置 [https://texdoc.org/serve/fancyhdr/0]
% (3){indentfirst}包:    控制段落的首行缩进 [https://texdoc.org/serve/indenfirst/0]
% (4){setspace}包:       控制行间距(非段落间距,段落间距由documentclass控制) [https://texdoc.org/serve/setspace/0]
% (5){fontspec}包:       自定义文档字体大小与样式,英文模板中不含此包,会报错 [https://texdoc.org/serve/fontspec/0]
% (6){tocloft}包:        控制目录格式,提供插图目录和表格目录 [https://texdoc.org/serve/tocloft/0]
% (7){chngcntr}包:       计数器,允许图、表等分章节重新编号 [https://texdoc.org/serve/chngcntr/0]   
% (8){lastpage}包:       允许调用页码 [https://texdoc.org/serve/lastpage/0]
% (9){amssymb}包:        提供LaTeX数学符号排版环境 [https://texdoc.org/serve/amssymb/0]
% (10){amsmath}包:       提供LaTeX数学公式排版环境 [https://texdoc.org/serve/amsmath/0]
% (11){xcolor}包:        为文档添加颜色 [https://texdoc.org/serve/xcolor/0]
% (12){colortbl}包:      为表格添加颜色 [https://texdoc.org/serve/colortbl/0]
% (13){diagbox}包:       制作斜线表头 [https://texdoc.org/serve/diagbox/0]
% (14){longtable}包:     制作可跨页的长表格 [https://texdoc.org/serve/longtable/0]
% (15){multirow}包:      控制合并单元格操作 [https://texdoc.org/serve/multirow/0]
% (16){makecell}包:      控制单元格换行,避免长文本溢出 [https://texdoc.org/serve/makecell/0]
% (17){booktabs}包:      用于绘制三线表 [https://texdoc.org/serve/booktabs/0]
% (18){rotating}包:      用于旋转表格 [https://texdoc.org/serve/rotating/0]
% (19){graphicx}包:      提供LaTeX插图环境 [https://texdoc.org/serve/graphicx/0]
% (20){float}包:         提供浮动体环境 [https://texdoc.org/serve/float/0]
% (21){caption}包:       允许设置图表标题 [https://texdoc.org/serve/caption/0]
% (22){subfig}包:        允许插入多张子图 [https://texdoc.org/serve/subfig/0]
% (23){chemformula}包:   快速输入化学方程式 [https://texdoc.org/serve/chemformula/0]   
% (24){chemfig}包:       快速输入化学结构式 [https://texdoc.org/serve/chemfig/0]
% (25){hyperref}包:      控制超链接格式 [https://texdoc.org/serve/hyperref/0]
% (26){listings}包:      允许排版各种编程语言的源代码 [https://texdoc.org/serve/listings/0]
% (27){algorithm2e}包:   允许排版算法和伪代码 [https://texdoc.org/serve/algorithm2e/0]
% (28){enumerate}包:     允许使用不同类型的有序列表 [https://texdoc.org/serve/enumerate/0]
% (29){enumitem}包:      允许自定义列表格式 [https://texdoc.org/serve/enumitem/0]
% (30){pdfpages}包:      允许在文档中插入PDF页面 [https://texdoc.org/serve/pdfpages/0]
% (31){natbib}包:        提供Nature期刊标准引用格式 [https://texdoc.org/serve/natbib/0]
% (32){lipsum}包:        生成随机文本,用于测试 [https://texdoc.org/serve/lipsum/0]
% (33){amsthm}包:        提供LaTeX数学定理环境 [https://texdoc.org/serve/amsthm/0]
% (34){amsfonts}包:      提供LaTeX数学字体 [https://texdoc.org/serve/amsfonts/0]
% -----------------宏包功能简介与使用手册------------------


% -----------------宏包引用------------------
\usepackage[left=1.5cm,right=1.5cm,top=2cm,bottom=1.5cm]{geometry}
\usepackage{fancyhdr}
\usepackage{indentfirst}
\usepackage{setspace}
\usepackage{tocloft}
\usepackage{chngcntr}
\usepackage{lastpage}
\usepackage{amssymb}
\usepackage{amsmath}
\usepackage{amsthm}
\usepackage{amsfonts}
\usepackage[table]{xcolor}
\usepackage{colortbl}
\usepackage{diagbox}
\usepackage{longtable}
\usepackage{makecell}
\usepackage{multirow}
\usepackage{booktabs}
\usepackage[figuresright]{rotating}
\usepackage{graphics}
\usepackage{float} 
\usepackage[font=small,labelfont=bf,labelsep=none]{caption}
\usepackage{subfig}
\usepackage{chemformula}
\usepackage{chemfig}
\usepackage[colorlinks,linkcolor=black]{hyperref}
\usepackage{listings}
\usepackage[ruled,vlined]{algorithm2e}
\usepackage{enumerate}
\usepackage{enumitem}
\usepackage{pdfpages}
\usepackage[compress]{natbib}
\usepackage{lipsum}

% -----------------宏包引用------------------
