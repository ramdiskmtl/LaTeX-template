% -------------------------------------------------------

%   LaTeX日常文件模板 by Meng Zhiran

%   introductory.tex : 导言区设置文件,设置编号规则、引用上标、页眉页脚、超链接等

% -------------------------------------------------------

% 目录设置为居中
\renewcommand{\cfttoctitlefont}{\hfill\Large\bfseries}
\renewcommand{\cftaftertoctitle}{\hfill}

% 章节格式设置
\ctexset{
  section={format={\songti\zihao{3}\bfseries}},
	subsection={format={\songti\zihao{4}\bfseries},beforeskip=0pt,afterskip=0pt},
	subsubsection={format={\songti\zihao{-4}},beforeskip=0pt,afterskip=0pt}
}

\captionsetup{labelsep=quad}% 设置图表标题与编号之间的距离

\counterwithin{figure}{section} % 让图编号分章节编号
\counterwithin{table}{section}  % 让表编号分章节编号
\numberwithin{equation}{section}% 让公式编号分章节编号

\renewcommand{\baselinestretch}{1.5} %行间距

\makeatletter
\newcommand\dlmu[2][5cm]{\hskip1pt\underline{\hb@xt@ #1{\hss#2\hss}}\hskip3pt} % 使用自定义的\dlmu命令创造指定长度的下划线,默认5cm
\makeatother


%设置数字、英文字体
\setmainfont{Times New Roman}

%设置目录字体和格式
\renewcommand{\cftsecfont}{\songti\bfseries}
\renewcommand{\cftsubsecfont}{\songti}
\renewcommand{\contentsname}{目\quad 录}

%设置参考文献格式
\renewcommand{\refname}{\heiti 参\ 考\ 文\ 献}

%设置\Emph命令为加粗红色
\newcommand{\Emph}[1]{\textbf{\color{red}#1}} 
%设置\emph命令为加下划线
\renewcommand{\emph}[1]{\underline{#1}} 

\hypersetup{hidelinks} %使超链接文本与周围文本无视觉差异

% \setlength{\abovecaptionskip}{0pt} %调整标题与图距离

% 页眉页脚设置
\pagestyle{fancy}

\fancyhf{}
\fancyhead[L]
	{
		\hspace{0cm}
		\begin{minipage}[b]{0.05\linewidth}
			\includegraphics[height=8mm]{Graph/sjtu_sign_cn.png}
		\end{minipage}
	}

\fancyhead[R]{\small{测试页眉1}}
\fancyfoot[C]{\thepage}

% 设置代码高亮颜色
\definecolor{dkgreen}{rgb}{0,0.6,0}
\definecolor{mauve}{rgb}{0.58,0,0.82}

% Python的代码风格设置
\lstdefinestyle{Python}{
  language=Python,
  basicstyle=\ttfamily\small,
  keywordstyle=\color{blue},
  commentstyle=\color{dkgreen},
  stringstyle=\color{mauve},
  breaklines=true,
  breakatwhitespace=true,
  numbers=left,
  numberstyle=\tiny\color{gray},
  numbersep=10pt,
  tabsize=3,
  frame=single
}

% R的代码风格设置
\lstdefinestyle{R}{
  language=R,
  basicstyle=\ttfamily\small,
  keywordstyle=\color{red},
  commentstyle=\color{olive},
  stringstyle=\color{purple},
  breaklines=true,
  breakatwhitespace=true,
  numbers=left,
  numberstyle=\tiny\color{gray},
  numbersep=10pt,
  tabsize=3,
  frame=single
}

% 将Reference居中
\def\bibsection{\centering\section*{\refname}}